\documentclass[10pt]{beamer}

\usepackage[T1]{fontenc}
\usepackage[utf8]{inputenc}
\usepackage{pgfpages}
\usepackage{xcolor}
\usepackage{xypic}
\usepackage{upgreek}
\usepackage{graphicx}
\usepackage{hyperref}
\usepackage{mathpartir}

\usepackage{palatino}
% \usepackage{xypic}
\usepackage{txfonts}
%\usepackage[llbracket,rrbracket]{stmaryrd}
\usepackage{pgfpages}

\mode<presentation>
% \Usetheme{Goettingen}
\usecolortheme{rose}
\usefonttheme{serif}
\setbeamertemplate{navigation symbols}{}

% Frame number
\setbeamertemplate{footline}[frame number]{}

% These slides also contain speaker notes. You can print just the slides,
% just the notes, or both, depending on the setting below. Comment out the want
% you want.

%\setbeameroption{hide notes} % Only slides
%\setbeameroption{show only notes} % Only notes
\setbeameroption{show notes on second screen=right} % Both

% Give a slight yellow tint to the notes page
\setbeamertemplate{note page}{\pagecolor{yellow!5}\insertnote}\usepackage{palatino}

%%%%%%%%%%%%%%%%%%%%%%%%%%%%%%%%%%%%%%%%%%%%%%%%%%
%%% MACROS

%% Typing judgements & rules
\newcommand{\emptyctx}{{\cdot}} % empty context
\newcommand{\typingrule}[2]{\infer{#1}{#2}}
\newcommand{\of}{\,{:}\,} % typing of a variable in a context
\newcommand{\ofLet}[3]{#1 \,{:}{=}\, #2 \,{:}\ #3} % typing of a variable in a context
\newcommand{\types}{\vdash} % the turnstile

\newcommand{\checkTy}{\leftsquigarrow}
\newcommand{\inferTy}{\rightsquigarrow}

\newcommand{\Type}{\mathsf{Type}}
\newcommand{\prd}[1]{\Uppi_{(#1)}\,}
\newcommand{\lam}[1]{\lambda (#1).\,}

\newcommand{\letin}[3]{\mathsf{let}\,#1\,{{:}{=}}\,#2\,\mathsf{in}\,#3}

\newcommand{\nfEquiv}{\mathrel{\overset{\scriptscriptstyle\mathrm{nf}}{\simeq}}}
%\newcommand{\neuEquiv}{\mathrel{\overset{\scriptscriptstyle\mathrm{neu}}{\simeq}}}

\newcommand{\norm}{\hookrightarrow}

%%%%%%% GROUP: named inference rules

% the style for rule names
\newcommand{\rulename}[1]{\textnormal{\textsc{#1}}}

% use \rref{...} to refer to a rule in text
\newcommand{\rref}[1]{\hyperlink{rule:#1}{\rulename{#1}}}

% the color of rule names
\definecolor{rulenameColor}{rgb}{0.5,0.5,0.5}

\definecolor{presupColor}{rgb}{0,0,1}
\definecolor{grayoutColor}{rgb}{0.7,0.7,0.7}

% named inference rule
\newcommand{\inferenceRule}[3]{\inferrule*[lab={\hypertarget{rule:#1}{\rulename{\footnotesize\color{rulenameColor}#1}}}]{#2}{#3}}


%%%%%%%%%%%%%%%%%%%%%%%%%%%%%%%%%%%%%%%%%%%%%%%%%%

\title{Programming language techniques \\for proof assistants\\[2ex]Lecture 2\\A monadic type checker}
\author{Andrej Bauer\\University of Ljubljana}
\date{}
\begin{document}

\begin{frame}
\hbox{}\vfil

\titlepage

\vfil

\begin{center}
\footnotesize
International School on Logical Frameworks and Proof Systems Interoperability \\
Université Paris--Saclay, September 8--12, 2025
\end{center}

\end{frame}


\begin{frame}
  \frametitle{Overview}

  \begin{itemize}
  \item \textcolor{grayoutColor}{Lecture 1: From declarative to algorithmic type theory}
  \item
    Lecture 2: A monadic type checker
    \begin{itemize}\footnotesize
    \item Parsing, bound variables and substitution
    \item A monad for typing contexts
    \item A monadic proof checker
    \end{itemize}
  \item
    \textcolor{grayoutColor}{Lecture 3: Holes and unification}
  \item
    \textcolor{grayoutColor}{Lecture 4: Variables as computational effects}
  \end{itemize}
\end{frame}

%%%%%%%%%%%%%%%%%%%%%%%%%%%%%%%%%%%%%%%%%%%%%%%%%%%%%%%%%%%%%%%%%%%%%%
%%%%%%%%%%%%%%%%%%%%%%%%%%%%%%%%%%%%%%%%%%%%%%%%%%%%%%%%%%%%%%%%%%%%%%
\begin{frame}
  \begin{center}
    \Huge Lecture 2

    \bigskip

    \Large
    A monadic type checker
  \end{center}
\end{frame}

\begin{frame}
  \frametitle{Recap}
\end{frame}


\begin{frame}
  \frametitle{Components of a proof checker}
\end{frame}

\begin{frame}
  \frametitle{OCaml?}
\end{frame}

\begin{frame}
  \frametitle{Solved problems: parsing}
\end{frame}

\begin{frame}
  \frametitle{Solved problems: bound variables and substitution}
\end{frame}

\begin{frame}
  \frametitle{Abstract syntax}

  show the data types from \texttt{TT}.
\end{frame}

\begin{frame}
  \frametitle{A monad for contexts}
\end{frame}

\begin{frame}
  \frametitle{Type-checking}
\end{frame}

\begin{frame}
  \frametitle{Normalization and equality}
\end{frame}


\end{document}
